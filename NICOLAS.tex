\documentclass[a4paper,12pt]{report}
\begin{document}
\title{\textbf{THE REPORT ON POVERTY IN UGANDA ACCORDING TO THE RESEARCH THAT WAS MADE IN 2016}}
\author{\textbf{NDAGGA NICHOLAS BRENDEN: 15/U/11200/EVE}}
\date{\textit{\textbf{7/04/2017}}}
\maketitle
\section{introduction}
{\Large Uganda has been experiencing record breaking poverty since the early years of 2000 causing thousands of people to be stranded.An estimate of about 31.1% has live in this poverty since the year 2000.Water and sanitation have dropped leading to numerpous death of peoplew in the poverty torn areas.
This forced the government to encourage people acquire some development loans from the commercial banks to mitigate on poverty.
But now the level of poverty has managed to drop from 31.1% in 2006 to 19.9% according to the report that was released in 2016.
Even though poverty has manged to drop to those levels progress in in reducing poverty has been much slower in northern and eastern parts of uganda and thud concentration of poverty is much in those areas 
The population of people living in poverty in these has increased form 64% to 84%.
Since the concentration over a decade was largely driven by the good fortune of the few hence leaving very many families vulnerable
In oder to achieve development uganda must take priority actions to fight poverty in the a sustainable manner.
The critical role that agriculture has played to reduce on poverty must be re examined with focus.
The government should also enforce industrialization has it has managed to transform many developing countries to development heme eliminating on poverty in uganda}
\section{Purpose}
{\Large The purpose of the survey was to find out the causes of poverty in the country
and devise means of mitigating the poverty to improve the lives of the people in uganda.}
\section{Scope}
{\Large The survey is tailored around poverty,sanitation,health,and development in Uganda.}
\section{Methods}
{\Large i carried out a person to person research in most of the poverty areas in Uganda and managed to acquire the following information.}
\section{Limitations}
{\Large The poor roads in most of these areas hindered the research making the transport cost high which hindered the movement
Poor communication due language barrier.}
\section{Findings}
{\Large I found out that poverty has greatly been caused by poor roads brought about poor government policies on development in the rural areas
Since uganda is an agriculture economy,most people in these areas are farmer and transportation has hinder movement of their produce to the markets.
This has created more damage in these areas leading to poor health,sanitation.}
\section{Conclusions}
{\Large poverty can be driven out of these areas if the government focuses on agricultural modernization since it contributes greatly to the GDP of the country with 98%}
\pagenumbering{roman}
\end{document}